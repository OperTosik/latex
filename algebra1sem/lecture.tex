\documentclass[a4paper, 12pt]{article}

\usepackage{cmap}
\usepackage[T2A]{fontenc}
\usepackage[english, russian]{babel}
\usepackage[utf8]{inputenc}
\usepackage[left=2cm,right=1.5cm,top=2cm,bottom=2cm]{geometry}
% \usepackage{mathtext}
\usepackage{amsmath}
\usepackage{amssymb}
\usepackage{etoolbox}
\usepackage{amsthm}
\usepackage{booktabs}
% \usepackage{nicematrix}
\usepackage{graphicx}
% \usepackage{tikz}
% \usepackage{parskip}

%Реализация aug, overbrace и underbrace без nice matrix
\newcommand\aug{\fboxsep=-\fboxrule\!\!\!\fbox{\strut}\!\!\!}
\newcommand\undermat[2]{\makebox[0pt][l]{$\smash{\underbrace
{\phantom{\begin{matrix}#2\end{matrix}}}_{\text{$#1$}}}$}#2}
\newcommand\overmat[2]{\makebox[0pt][l]{$\smash{\overbrace
{\phantom{\begin{matrix}#2\end{matrix}}}^{\text{$#1$}}}$}#2}
\newcommand\tab[1][.5cm]{\hspace*{#1}}
\newcommand\Underset[2]{\underset{\textstyle #1}{#2}}
\newcommand\Overset[2]{\overset{\textstyle #1}{#2}}


\theoremstyle{definition}
\newtheorem*{definition}{Определение}
\newtheorem*{theorem}{Теорема}
\newtheorem*{consequense}{Следствие}
\newtheorem*{lemma}{Лемма}
\newtheorem*{subtheorem}{Утверждение}
\newtheorem*{remark}{Замечание}

\usepackage[russian]{babel}
\addto\captionsenglish{% Replace "english" with the language you use
  \renewcommand{\contentsname}%
    {Содержание}%
}

\usepackage{titlesec}
\titleformat{\section}{\LARGE \bfseries}{\thesection}{1em}{}
\titleformat{\subsection}{\Large\bfseries}{\thesubsection}{1em}{}
\titleformat{\subsubsection}{\large\bfseries}{\thesubsubsection}{1em}{}

\usepackage{hyperref}
\usepackage{xcolor}
% Цвета для гиперссылок
\definecolor{linkcolor}{HTML}{225ae2} % цвет ссылок
\definecolor{urlcolor}{HTML}{225ae2} % цвет гиперссылок
\hypersetup{
    pdfstartview=FitH, 
    linkcolor=linkcolor,
    urlcolor=urlcolor,
    colorlinks=true
}

\title{\textbf{Алгебра. 1 семестр, Куликова}}
\author{Ким Никита, 211 группа}

\begin{document}
\fontsize{14pt}{20pt}\selectfont
    \maketitle
    \newpage
    \tableofcontents
    \fontsize{14pt}{20pt}\selectfont
    \newpage
    
    \section{Системы линейных уравнений}
    \subsection{Матрица. Освноные понятия}
    \begin{definition}
        Матрица $A$ размера $m\times n$ --   прямоугольная таблица с $m$ строками и $n$ столбцами 
        $$A = \begin{pmatrix}
            a_{11} && a_{12} && \dots && a_{1n}\\
            \vdots && \null && \null && \vdots\\
            a_{n1} && a_{n2} && \null && a_{nn}
        \end{pmatrix}$$
        $a_{ij}$ -- элемент матрицы и индексы:
        
        $i$ -- номер строки
        
        $j$ -- номер столбца\\
        $M_{m\times n}(\mathbb{R})$ -- множество всех матриц размера $m\times n$ c элементами из $\mathbb{R}$.
    \end{definition}
    Матрица размера $m\times 1$  называется столбцом $$A = \begin{pmatrix}
        a_{11}\\ \vdots\\ a_{m1}
    \end{pmatrix}$$
    Если $A = (a_{ij})$ -- квадратная, $a_{ij} = 0\ \forall i \neq j$, то $A$ называется диагональной.
    $$A = \begin{pmatrix}
        a_{11} && \null && \null && 0\\
        \null && a_{22} && \null && \null\\
        \null && \null && \ddots && \null\\
        0 && \null && \null && a_{nn}
    \end{pmatrix}$$   \\
    Если $A$ -- диагональная и $a_{ii} = 1$, то $A$ называется единичной.
    $$A = \begin{pmatrix}
        1 && \null && \null && 0\\
        \null && 1 && \null && \null\\
        \null && \null && \ddots && \null\\
        0 && \null && \null && 1
    \end{pmatrix}$$
    Если  $A$ -- квадратная, то 
    $$A = \begin{pmatrix}
        a_{11} && \null && \null && \null\\
        \null && a_{22} && \null && \null\\
        \null && \null && \ddots && \null\\
        \null && \null && \null && a_{nn}
    \end{pmatrix}$$ 
    главная диагональ, а 
    $$A = \begin{pmatrix}
        \null && \null && \null && a_{1n}\\
        \null && \null && a_{2,n-1} && \null\\
        \null && \ddots && \null && \null\\
        a_{n1} && \null && \null && \null
    \end{pmatrix}$$
    побочной диагональю.
    \begin{definition}
        Если $A$ -- размера $m\times n$, $a_{ij} = 0\ \forall i,j$, то $A$ называется нулевой.   
    \end{definition}
    \subsection{Система линейных (алгебраических) уравнений}
    $(*)\begin{cases}
        a_{11}x_1 + ... + a_{1n}x_n = b_1\\
        a_{21}x_1 + ... + a_{2n}x_n = b_2\\
        \vdots\\
        a_{n1}x_1 + ... + a_{nn}x_n = b_n
    \end{cases}$\\$\\$
    $a_{ij}, b \in \mathbb{R}, x_1,...,x_n$ -- неизвестные.
    $$A = \begin{pmatrix}
        a_{11} && \dots && a_{1n}\\
        \vdots && \null && \vdots\\
        a_{n1} && \dots && a_{nn}
    \end{pmatrix}\tab[1cm]B = \begin{pmatrix}
        b_{1}\\\vdots\\b_m
    \end{pmatrix}$$
    $A$ -- матрица коэффициентов, $a_{ij}$ называется коэфициентом СЛУ.\\
    $B$ -- столбец свободных членов, $b_j$ -- свободный член.
    \begin{definition}
        Расширенная матрица $\underset{m\times (n+1)}{(A|B)}$. Набор чисел $x_1^0,...,x_n^0 \in \mathbb{R}$ называется решением системы $(*)$, если подстановка этих чисел вместо неизвестных в $(*)$ дает тождество в каждом уравении. ($x_i^0 \longleftrightarrow x$) 
    \end{definition}
    Решить систему -- это найти все решения системы. Любое конткретное решение называется частным.

    \begin{definition}
        Если СЛУ имеет решение, то она называется совместной, иначе несовместной.
    \end{definition}
    \begin{definition}
        Совместная система, имеющая одно решение называется определенной, иначе неопределенной (решений более одного).
    \end{definition}
    \subsection{Элементарные преобразования над СЛУ}
    \begin{enumerate}
        \item Прибавить к одному уравнению другое уравнение умноженное на число $\lambda \in \mathbb{R}$ 
        \item Поменять местами два уравнения
        \item Можем умножить уравненние на ненуленое число $\mu \in \mathbb{R}$ 
    \end{enumerate}
    \begin{subtheorem}
        Эти преобразования обратимы.
    \end{subtheorem}
    \begin{definition}
        Две системы линейных уравнений называюся эквивалентными, если их множество решений совпадают.
    \end{definition}
    \begin{subtheorem}
        Если одна СЛУ получена из другой СЛУ с помощью конечного числа элементарных преобразований, то эти системы эквивалентны.
    \end{subtheorem}
    \begin{proof}
        Достаточно доказать, что если $AX = B$ и $\widetilde{A}X = \widetilde{B}$ получены с помощью одного элементарного преобразования, то они эквивалентны.\\
        Пусть $x_1^0,...,x_n^0$ -- произвольное решение $AX = B$.\\
        Докажем, что $x_1^0,...,x_n^0$ является решением системы $\widetilde{A}X = \widetilde{B}$.\\
        Для 2 пункта очевидно.
        Для 3: предположим, что $$a_{i1}x_1 +...+ a_{in}x_n = b_i \text{ в } AX = B$$
        $$(\mu a_{i1})x_1 +...+ (\mu a_{in})x_1n = \mu b_i \text{  в } \widetilde{A}X = \widetilde{B}$$
        Если $a_{i1}x_1 +...+ a_{in}x_n = b_i$, то $(\mu a_{i1})x_1 +...+ (\mu a_{in})x_1n = \mu b_i$.\\
        Остальные уравнения такие же $\mu (a_{i1}x_1 +...+ a_{in}x_n) = (\mu)b_i$\\
        1 пункт: д/з.\\
        Т.о. множество решений $AX = B$ включено в множество решений $\widetilde{A}B = \widetilde{B}$.\\
        В обратную сторону аналогично (для доказательства эквивалентности), используя обратимость элементарных преобразований.
    
    \end{proof}
    Мораль в том, что мы можем работать с расширенной матрицей $(A|B)$.
    \subsection{Элементарное преобразование над матрицами}
    \textbf{Элементраные преобразования над строкаим матрицы:}
    $$A = \begin{pmatrix}
        \overline{a_1}\\\vdots\\\overline{a_n}
    \end{pmatrix},\text{ где } \overline{a_i} - \text{ строка}$$
    \begin{itemize}
        \item ЭП1: $\overline{a_i} \to \overline{a_i} + \lambda\overline{a_j}$
        \item ЭП2: $\overline{a_i} \longleftrightarrow \overline{a_j}$ 
        \item ЭП3: $\overline{a_i} \to \mu \overline{a_i},\ \mu \neq 0$ 
    \end{itemize}
    \begin{definition}
        Лидер строки (ведущий элемент) -- это 1-й ненулевой элемент слева.\\
        \textbf{Пример:}
        $(0, 0, \underbrace{3}_{\text{лидер}}, 4, 5, 0, 0, 7)$ 
    \end{definition}
    \begin{definition}
        Матрица $A$ размера $m\times n$ называется ступенчатой, если
        \begin{enumerate}
            \item Номера лидеров ненулевых строк строго возврастают с увеличением номера строки.
            \item Все нулевые строки стоят внизу (в конце).
        \end{enumerate}
    \end{definition}
    \begin{theorem}
        Любую матрицу $A$ размера $m\times n$ за конечное число элементарных преобразований над строками можно привести к ступенчатому виду.
    \end{theorem}
    \begin{proof}
        Индукция по $n:$\\
        Если $A$ -- нулевая, то $A$ -- ступенчатого вида.ёё
        Если $A \neq 0:$ найдем первый ненулевой столбец (начиная слева). Пусть $j$ -- номер первого ненулевого столбца. Пусть $a_{ij} \neq 0:$
        $$A = \begin{pmatrix}
            0 && 0 && && \null\\
            \vdots && \vdots && \null\\
            \null && \null && a_{ij}\\
            \vdots && \vdots && \null\\
            0 && 0 && \null
        \end{pmatrix}$$  
        Меняем 1-ю и i-ю строки местами и получаем, что $a_{ij}$ стал лидером первой строки. Считаем, что сразу $a_{1j} \neq 0:$
        $$A = \begin{pmatrix}
            0 && 0 && a_{1j} && *\\
            \vdots && \vdots && * && *\\
            \null && \null && \vdots && \null\\
            \vdots && \vdots && \vdots && \null\\
            0 && 0 && \vdots && \null
        \end{pmatrix}$$  
        Вычитаем из каждой k-й строки, начиная со 2-й 1-ю строку, умноженную на число $\frac{a_{kj}}{a_{1j}}$. Получаем вид:
        $$\widetilde{A} = \begin{pmatrix}
            0 && 0 && \vline && *\\
            \vdots && \vdots && \vline &&  * && *\\
            \null && \null && \vline && \vdots && \null\\
            \vdots && \vdots && \vline && \vdots && \null\\
            0 && 0 && \vline &&  \vdots && \null
        \end{pmatrix}$$ 
        К правой части матрицы применяем и ндукцию и приводим матрицу к ступенчатому виду.

    \end{proof}
    \begin{remark}
        Этот метод называется методом Гаусса.
    \end{remark}
    

\end{document}